\chapter{Translating CST to AST}

\section{Converting literals}

How a CST corresponding to a literal in source code is converted,
depends on whether the conversion is using file-compilation semantics
of \texttt{eval} semantics.  When \texttt{eval} semantics is used, the
literal is converted to a \texttt{literal-ast} with the literal in a
slot in the AST.  The remainder of this section discusses how the
conversion is done when file-compilation semantics is used.

We keep an \texttt{eql} hash table, mapping source literals to
\emph{literal entries}.  Each literal entry contains the following
information:

\begin{itemize}
\item The source literal itself for error reporting.
\item Two \emph{AST entries}, one corresponding to the creation form
  (called the ``creation entry'') and the other to the initialization
  form (called the ``initialization entry'').
\end{itemize}

An AST entry contains the following information:

\begin{itemize}
\item The source form to be converted to an AST.
\item The AST that, when evaluated, creates or initializes the
  literal.  When the entry is initially created, this information is
  not available.  It is generated as a result of the conversion of the
  source form.
\item A list of AST entries that this entry depends on.  When one AST
  entry depends on another AST entry, this means that the code of the
  latter must be executed before the code of the former.  Every AST
  entry in this list is a creation entry.
\item When the AST entry is a creation entry, a list of entries that
  depend on this entry.  An entry in this list may be either a
  creation entry or an initialization entry.
\item When the AST entry is a creation entry, a \emph{lexical-ast}
  that holds the literal, once it has been created from by the
  \emph{creation AST}.
\end{itemize}

We also keep a \emph{work list} of AST entries that have not yet had
their source forms converted to ASTs.

Converting a literal is done as follows:

\begin{itemize}
\item The hash table is consulted to see whether the literal has
  already been converted.
\item If the literal has not already been converted, a literal entry
  is created and added to the table.  A call is made to
  \texttt{make-load-form-using-client} in order to obtain the creation
  form and the initialization form.  These forms are used to create
  the two AST entries in the literal entry.  A \texttt{lexical-ast} is
  created and stored in the creation entry.  The initialization entry
  is added to the list of AST entries depending on the creation entry.
  The creation entry is added to the list of entries that the
  initialization entry depends on. The two AST entries are added to
  the work list.
\item The special variable \texttt{*current-ast-entry*} is consulted.
  If its value is \texttt{nil}, no action is taken.  Otherwise, its
  value indicates the AST entry containing the form that is currently
  being converted, and call it $A$.  If so, $A$ is added to the list
  of AST entries depending on the creation entry of the hash table
  literal entry, and the creation entry of the hash table literal
  entry is added to the list of AST entries $A$ depends on.  The value
  of the special variable \texttt{*current-ast-entry*} is \texttt{nil}
  when the body of the compilation unit is processed, and bound as
  described below.
\item The result of the conversion is the \texttt{lexical-ast} stored
  in the AST entry corresponding to the creation form.
\end{itemize}

When the conversion of the entire compilation unit is finished, we
then process the entries in the work list.  Processing an entry
consists of binding the special variable \texttt{*current-ast-entry*}
to the entry and then converting the source form to an AST.  This step
may result in new literal entries and new AST entries to be created
and entered into the table and work list respectively.  Processing
continues until the work list is empty.

In the last step, we maintain a work list as a LIFO.

\begin{itemize}
\item When the work list is empty and there are no more AST entries to
  process, we are done.
\item Otherwise, if the work list is empty, but there are more AST
  entries to process, we find an AST entry that does not depend on any
  other entry.  If no such entry can be found, there is a circular
  dependency between creation forms and an error is signaled.  If an
  entry can be found, it is pushed onto the work list.
\item Otherwise, there is an entry on the work list, say $A$.  The AST
  of $A$ is then entered into the prologue of the AST corresponding to
  the compilation unit, and $A$ is eliminated as a dependency of all
  other AST entries. If any AST entry $B$ depending on $A$ depends
  only on $A$, then $B$ is pushed onto the LIFO.  The reason for the
  use of a LIFO is to respect the rule stated in the standard that an
  initialization form should be executed ``as soon as possible'' after
  the corresponding creation form.
\end{itemize}

