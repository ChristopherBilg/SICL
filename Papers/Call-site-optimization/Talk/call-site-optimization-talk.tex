\documentclass{beamer}
\usepackage[utf8]{inputenc}
\beamertemplateshadingbackground{red!10}{blue!10}
%\usepackage{fancybox}
\usepackage{epsfig}
\usepackage{verbatim}
\usepackage{url}
%\usepackage{graphics}
%\usepackage{xcolor}
\usepackage{fancybox}
\usepackage{moreverb}
%\usepackage[all]{xy}
\usepackage{listings}
\usepackage{filecontents}
\usepackage{graphicx}

\lstset{
  language=Lisp,
  basicstyle=\scriptsize\ttfamily,
  keywordstyle={},
  commentstyle={},
  stringstyle={}}

\def\inputfig#1{\input #1}
\def\inputeps#1{\includegraphics{#1}}
\def\inputtex#1{\input #1}

\inputtex{logos.tex}

%\definecolor{ORANGE}{named}{Orange}

\definecolor{GREEN}{rgb}{0,0.8,0}
\definecolor{YELLOW}{rgb}{1,1,0}
\definecolor{ORANGE}{rgb}{1,0.647,0}
\definecolor{PURPLE}{rgb}{0.627,0.126,0.941}
\definecolor{PURPLE}{named}{purple}
\definecolor{PINK}{rgb}{1,0.412,0.706}
\definecolor{WHEAT}{rgb}{1,0.8,0.6}
\definecolor{BLUE}{rgb}{0,0,1}
\definecolor{GRAY}{named}{gray}
\definecolor{CYAN}{named}{cyan}

\newcommand{\orchid}[1]{\textcolor{Orchid}{#1}}
\newcommand{\defun}[1]{\orchid{#1}}

\newcommand{\BROWN}[1]{\textcolor{BROWN}{#1}}
\newcommand{\RED}[1]{\textcolor{red}{#1}}
\newcommand{\YELLOW}[1]{\textcolor{YELLOW}{#1}}
\newcommand{\PINK}[1]{\textcolor{PINK}{#1}}
\newcommand{\WHEAT}[1]{\textcolor{wheat}{#1}}
\newcommand{\GREEN}[1]{\textcolor{GREEN}{#1}}
\newcommand{\PURPLE}[1]{\textcolor{PURPLE}{#1}}
\newcommand{\BLACK}[1]{\textcolor{black}{#1}}
\newcommand{\WHITE}[1]{\textcolor{WHITE}{#1}}
\newcommand{\MAGENTA}[1]{\textcolor{MAGENTA}{#1}}
\newcommand{\ORANGE}[1]{\textcolor{ORANGE}{#1}}
\newcommand{\BLUE}[1]{\textcolor{BLUE}{#1}}
\newcommand{\GRAY}[1]{\textcolor{gray}{#1}}
\newcommand{\CYAN}[1]{\textcolor{cyan }{#1}}

\newcommand{\reference}[2]{\textcolor{PINK}{[#1~#2]}}
%\newcommand{\vect}[1]{\stackrel{\rightarrow}{#1}}

% Use some nice templates
\beamertemplatetransparentcovereddynamic

\newcommand{\A}{{\mathbb A}}
\newcommand{\degr}{\mathrm{deg}}

\title{Call-site optimization for \commonlisp{}}

\author{Robert Strandh}
\institute{
LaBRI, University of Bordeaux
}
\date{April, 2021}

%\inputtex{macros.tex}


\begin{document}
\frame{
\titlepage
\vfill
\small{European Lisp Symposium, Online Everywhere \hfill ELS2021}
}

\setbeamertemplate{footline}{
\vspace{-1em}
\hspace*{1ex}{~} \GRAY{\insertframenumber/\inserttotalframenumber}
}

\frame{
\frametitle{Context: The \sicl{} project}

https://github.com/robert-strandh/SICL

Several objectives:

\begin{itemize}
\item Create high-quality \emph{modules} for implementors of
  \commonlisp{} systems.
\item Improve existing techniques with respect to algorithms and data
  structures where possible.
\item Improve readability and maintainability of code.
\item Improve documentation.
\item Ultimately, create a new implementation based on these modules.
\end{itemize}
}

\frame{
  \frametitle{Context for this work}
  Cost of function calls in \commonlisp{}:
  \begin{itemize}
  \item Late binding.
  \item Optional parameters and keyword parameters.
  \item Callee must often check types of arguments.
  \item Arguments must be boxed.
  \item Generic functions can be dynamically updated.
  \item Multiple return values complicate caller.
  \end{itemize}
}

\frame{
  \frametitle{Late binding}
  Usually handled by an indirection through a symbol.
  \vskip 0.25cm
  In SICL, handled by an indirection through a \emph{function cell} in
  a first-class global environment.
  \vskip 0.25cm
  Late binding is no longer respected in the presence of compiler
  macros or inlining.
}

\frame{
  \frametitle{Optional parameters and keyword parameters}
  Callee must at least check the argument count.
  \vskip 0.25cm
  With optional parameters, a conditional branch is required.
  \vskip 0.25cm
  Keyword parameters require iterating over remaining arguments.
}

\frame[containsverbatim]{
\frametitle{blabla}
\begin{verbatim}
(defgeneric foo (x)
  (:method-combination and :most-specific-last))
\end{verbatim}
}

\frame{
\frametitle{Previous work}
We investigated:
\begin{itemize}
\item cmucl
\item Steel Bank Common Lisp (SBCL)
\item Clasp
\item Allegro
\end{itemize}
}

\frame{
\frametitle{Our technique}
\vskip 0.25cm
%% \begin{figure}
%% \begin{center}
%% \inputfig{fig-method-combinations.pdf_t}
%% \end{center}
%% \end{figure}
}

\frame{
\frametitle{Future work}
blabla
\vskip 0.25cm
blabla
}

\frame{
  \frametitle{Acknowledgments}

We would like to thank...
}

\frame{
\frametitle{Thank you}
}

%% \frame{\tableofcontents}
%% \bibliography{references}
%% \bibliographystyle{alpha}

\end{document}
