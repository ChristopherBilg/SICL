
as promised, here are the instructions on how you can format your
paper correctly for the proceedings.  The following code should be in
the preamble of your TeX-file: 

-----------
\documentclass[format=sigconf]{acmart}
\bibliographystyle{plainnat}
\acmConference[ELS’21]{the 14th European Lisp Symposium}{May 03--04 2021}{Online, Everywhere}
\acmDOI{10.5281/zenodo.XXXXXXX}
\setcopyright{rightsretained}
\copyrightyear{2021}
-----------

You can obtain your DOI (the missing XXXXXXX in the above TeX-sippet)
via Zenodo.  The process is a bit involved, but has the wonderful
benefit that your paper is archived for all time and can be found
easily. I cannot do this process for you, but at least I can provide
detailed instructions: 

1. Go to https://zenodo.org/communities/els

2. Click on "New Upload" (should be a big, green button)

3. You should see a form for uploading a new document.  Skip the first
tab for uploading files, and fill in all the metadata.

4. In the 'Communities' tab, pick 'European Lisp Symposium'

5. In the 'Upload type' tab, pick 'Publication', and select the
'Conference paper' publication type.

6. The 'Basic information' tab will get you your DOI.  Do not fill in
the 'Digital Object Identifier' field, but click the 'Reserve DOI'
button below it.  Zenodo will then assign you a DOI.  Include this DOI
in your paper and recompile it.  Then upload your finished paper in
the 'Files' tab.  (This process resolves the circular dependency of
the DOI being included in the file it points to.).

7. Fill out the remaining items in the 'Basic information' tab.  The
'Publication date' is 2021-05-03, the 'Title' should be the title of
your paper, the 'Authors' tab should include each author and
affiliation (if you plan to publish more papers in the future, I also
recommend you register an ORCID for yourself and put it there), and
the 'Description' should be the abstract of your paper.  The other
fields are not so important - if you want you can declare your paper's
language as 'eng', and add your paper's keywords to the 'Keywords'.

8. In the 'License' tab, select 'Open Access', and pick a license.  If
you are not sure about which license to pick, select 'Creative Commons
Attribution No Derivatives 4.0 International'.

9. In the 'Related/alternative identifiers' tab, you should specify
that the paper is part of the ELS proceedings by adding an entry with
value '2677-3465' with the property 'is compiled/created by this
upload'. Leave the resource type as 'N/A'.

10. In the 'Conference' tab, set the 'Conference title' to 'The 14th
European Lisp Symposium', the 'Acronym' to 'ELS'21', the 'Dates' to
3-4 May 2021, the 'Place' to 'Online', and the website to
'https://european-lisp-symposium.org/2021'.

11. Double-check that all fields have been filled out correctly and
that you attached the correct file.  Then click 'Publish'.

Once this is done, upload the final version of that paper on Easychair
and your work as authors is complete.  Which means you should start
thinking about your ELS talk (if you haven't done so already).  I also
recommend you start experimenting with your recording setup right
away.  It would be a shame if your talk would be hampered by something
like bad audio quality.
