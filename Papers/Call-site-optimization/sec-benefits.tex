\section{Benefits of our technique}
\label{sec-benefits}

Our technique makes possible several features that are not possible
when the function calls are created by the caller, without knowledge
about the callee.

For starters, at least one indirection can be avoided, thereby saving
a memory access.  When the call is generated by the caller, there must
be an indirection though some kind of \emph{function cell}, unless the
callee is a function that is known never to change.  This indirection
is required so that a redefinition of the callee is taken into account
by the next call.  A typical \commonlisp{} implementation uses a
symbol (the name of the function) for this indirection, whereas
\sicl{} uses a separate \texttt{cons} cell, but the cost is the same.
With our technique, when a callee is altered, the snippet is
modified.  As a result, no indirection is required.  Furthermore, in
\sicl{} all functions are standard objects which requires another
indirection from the \emph{header object} to the so-called \emph{rack}
where the entry point is stored.

Furthermore, argument parsing can be greatly simplified.  Even in the
simple case where all parameters are required, it is no longer
necessary for the caller to pass the argument count, nor for the
callee to check that it corresponds to the number of parameters.  But
the advantages are even greater in the presence of optional parameters
and in particular for keyword parameters.  In a typical call with
keyword parameters, the keywords are literals.  The argument list can
then be parsed once and for all when the snippet is created, and the
arguments can be directly copied to the locations required by the
callee.  This possibility largely eliminates the need for separate
compiler macros, as the purpose of a compiler macro is precisely to
take advantage of some known structure of the list of argument, in
order to substitute a call to a specialized version of the callee.

The specialized function call can admit unboxed arguments.  Avoiding
boxing is particularly useful for applications that manipulate
floating-point values that are at least the size of the machine word,
say IEEE double our quadruple floats in a 64-bit system.  Using a
general-purpose function-call protocol, each such argument must be
encapsulated in a memory-allocated object before the call, and often,
the argument will immediately be unboxed by the callee for further
processing.

Return values benefit from the same advantages as arguments.  Often,
the number of values required by the caller is known statically.  The
callee can then specialize the transfer of those values to the right
locations in the caller.  And if the caller requires fewer values than
the callee computes, the callee can sometimes be specialized so that
extraneous return values do not need to be computed at all.  As with
arguments, return values can be unboxed, again avoiding costly memory
allocations.

Often, \emph{inlining} is used to improve the performance of function
calls, either by the application programmer or by the system itself.
But inlining some function necessarily increases the code size of each
caller of that function.  Furthermore, the semantics of inlining are
such that the caller must be recompiled for a modified callee to be
taken into account.  Our technique can often provide enough
performance improvement to make inlining unnecessary.  Total code size
will then be smaller, and the disadvantage of inlining with respect to
calle redefinition is eliminated.
