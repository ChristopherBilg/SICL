\section{Our technique}
\label{sec-our-technique}

For the purpose of this work, we define a \emph{function call} to be
the code that accomplishes the following tasks:

\begin{enumerate}
\item It accesses the arguments to be passed to the callee from the
  places they have been stored after computation, and puts the
  arguments in the places that the callee expects them.
\item It accesses the function object associated with the name at the
  call site and stores it in some temporary location.
\item From the function object, it accesses the static environment to
  be passed to the code of the callee.
\item Also from the function object, it accesses the \emph{entry
  point} of the function, i.e., typically the address of the first
  instruction of the code of the callee.
\item It transfers control to the entry point, using an instruction
  that saves the return address for use by the callee to return to the
  caller.
\item Upon callee return, it accesses the return values from the places
  they have been stored by the callee, and puts those values in the
  places where the caller requires them for further computation.
\end{enumerate}

In a typical implementation, a function call is generated when the
code of the caller is compiled, and it then never changes.  For this
permanent code to work, a particular \emph{call protocol} must be
observed, and that protocol must be independent of the callee, as the
callee may change after the caller has been compiled.

With our suggested method, the function call is created by the callee.
The code emitted by the caller for a function call consists of a
single unconditional \emph{jump} instruction.  The target address in
that instruction is altered by the callee according to its structure.
The code for the function call is contained in an object that we call
a \emph{trampoline snippet}, or just \emph{snippet} for short.  The
callee allocates an appropriate snippet in the global heap as describe
in \refSec{sec-sicl-features}, at some available location, and the
unconditional jump instruction of the caller is modified so that it
performs a jump to the first instruction of the snippet.  The
constellation of caller, callee, and snippet is illustrated in
\refFig{fig-snippet}.  We omitted an explicit indication of a control
transfer from the snippet to the callee code, because such a control
transfer is not always required.

\begin{figure}
\begin{center}
\inputfig{fig-snippet.pdf_t}
\end{center}
\caption{\label{fig-snippet}
Caller, callee, and snippet in the global heap.}
\end{figure}

When the callee changes in some way, a
new snippet is allocated and the jump instruction is altered to refer
to the position of the new snippet.  The old snippet is then subject
to garbage collection like any other object.

When code containing a caller is loaded into the global environment,
and that caller contains a call site that refers to a function that is
not defined at the time the caller is loaded, a \emph{default snippet}
is created.  The default snippet contains the same instructions that a
traditional compiler would create for a call to a function that might
be redefined in the future.  Thus, the default snippet contains code
to put parameters in places dictated by the calling conventions, and
it accesses return values from predefined places.  It also accesses
the function indirectly, either through a symbol object (as most
\commonlisp{} systems probably do) or through a separate
\emph{function cell} as described in our paper on first-class global
environments \cite{Strandh:2015:ELS:Environments}.  The default
snippet is illustrated in \refFig{fig-default-snippet}.  The default
snippet is also used when the definition of the callee changes, as
described below.  A default snippet for each call site can either be
kept around, or allocated as needed.  The former situation is
advantageous for a callee with many call sites and for callees that
are frequently redefined, as it decreases the time to load a new
version of the callee.

\begin{figure}
\begin{center}
\inputfig{fig-default-snippet.pdf_t}
\end{center}
\caption{\label{fig-default-snippet}
Default snippet.}
\end{figure}

In order for the callee to be able to adapt the snippet to its
requirements, the caller, when loaded into the executing image, must
provide information about its call sites to the system.  Each call
site contains information such as:

\begin{itemize}
\item The name of the callee.
\item The number of arguments.
\item The type of each argument.  If the type is not known, it is
  indicated as \texttt{t}.  When an argument is a literal object, its
  type is indicated as \texttt{(eql ...)}.
\item For each argument, whether the argument is boxed or unboxed.
\item For each argument, its location.  The location can be a register
  or a stack position in the form of an offset from a frame pointer.
\item The number of required return values, or an indication that all
  return values are required, no matter the number.
\item In case of a fixed number of return values, for each such value,
  whether it should be boxed or unboxed.
\item Also, in case of a fixed number of return values, for each such
  value, the location where the caller expects the value.
\item Indication as to whether the call is a \emph{tail call}, in
  which case the snippet should deallocate the frame before
  returning.
\end{itemize}

A callee can take advantage of this information to customize the
call.  The default action is to generate a snippet that implements the
full function-call protocol, without taking into account information
about the types of the arguments.

When a modification is made to a callee that alters its semantics,
care must be taken so as to respect the overall semantics of all
callers.  In particular, a callee can be removed using
\texttt{fmakunbound} or entirely replaced using \texttt{(setf
  fdefinition)}.  In that case, the following steps are taken in
order:

\begin{enumerate}
\item First, a default snippet is allocated for each caller, or the
  kept default snippet is reused.  The unconditional jump instruction
  is modified to refer to the default snippet.  As previously
  explained, this snippet contains code for the full function-call
  protocol, and in particular, it accesses the callee using an
  indirection through the function cell.
\item Next, the callee is atomically replaced by a new function, or
  entirely removed by a single modification to the contents of the
  function cell.
\item The new function is attached to the list of call sites, and,
  depending on the nature of the new function, new snippets can then
  be allocated in order to improve performance of calls to the new
  function.
\end{enumerate}

The thread responsible for redefining the callee blocks until step 1
is accomplished.  Without this blocking, some callers may get the old
version of the callee and some others the new version, thereby
violating the overall semantics of a function redefinition.

Step 3, on the other hand can be accomplished asynchronously, and even
in parallel with caller threads, provided that appropriate
synchronization prevents subsequent simultaneous redefinitions of the
callee.

%%  LocalWords:  callee redefinitions
