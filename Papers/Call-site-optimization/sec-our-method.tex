\section{Our technique}
\label{sec-our-technique}

For the purpose of this work, we define a \emph{function call} to be
the code that accomplishes the following tasks:

\begin{enumerate}
\item It accesses the arguments to be passed to the callee from the
  places they have been stored after computation, and puts the
  arguments in the places that the callee expects them.
\item It accesses the function object associated with the name at the
  call site and stores it in some temporary location.
\item From the function object, it accesses the static environment to
  be passed to the code of the callee.
\item Also from the function object, it accesses the \emph{entry
  point} of the function, i.e., typically the address of the first
  instruction of the code of the callee.
\item It transfers control to the entry point, using an instruction
  that saves the return address for use by the callee to return to the
  caller.
\end{enumerate}

In a typical implementation, a function call is generated when the
code of the caller is compiled, and it then never changes.  For this
permanent code to work, a particular \emph{call protocol} must be
observed, and that protocol must be independent of the callee, as the
callee may change after the caller has been compiled.

With our suggested method, the function call is created by the
callee.
