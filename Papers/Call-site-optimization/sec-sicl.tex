\section{Main features of the \sicl{} system}
\label{sec-sicl-features}

\sicl{} is a system that is written entirely in \commonlisp{}.  Thanks
to the particular bootstrapping technique
\cite{durand_irene_2019_2634314} that we developed for \sicl{}, most
parts of the system can use the entire language for their
implementation.  We thus avoid having to keep track of what particular
subset of the language is allowed for the implementation of each
module.

We have multiple objectives for the \sicl{} system, including
exemplary maintainability and good performance.  However, the most
important objective in the context of this paper is the design of the
garbage collector.  Our design is based on that of a concurrent
generational collector for the ML language \cite{Doligez:1993:CGG}.
We use a \emph{nursery} generation for each thread, and a global heap
for shared objects.  For the purpose of the current work, the
important feature of the garbage collector is that the objects in the
global heap do not move, and that all executable code is allocated in
that global heap.

The fact that code does not move is beneficial for the instruction
cache; moreover it crucially allows us to allocate different objects
in the global heap containing machine instructions, and to use fixed
relative addresses to refer to one such object from another such
object.
