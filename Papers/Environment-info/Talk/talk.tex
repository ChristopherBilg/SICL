\documentclass{beamer}
\usepackage[latin1]{inputenc}
\beamertemplateshadingbackground{red!10}{blue!10}
%\usepackage{fancybox}
\usepackage{epsfig}
\usepackage{verbatim}
\usepackage{url}
%\usepackage{graphics}
%\usepackage{xcolor}
\usepackage{fancybox}
\usepackage{moreverb}
%\usepackage[all]{xy}
\usepackage{listings}
\usepackage{filecontents}
\usepackage{graphicx}

\lstset{
  language=Lisp,
  basicstyle=\scriptsize\ttfamily,
  keywordstyle={},
  commentstyle={},
  stringstyle={}}

\def\inputfig#1{\input #1}
\def\inputeps#1{\includegraphics{#1}}
\def\inputtex#1{\input #1}

\inputtex{logos.tex}

%\definecolor{ORANGE}{named}{Orange}

\definecolor{GREEN}{rgb}{0,0.8,0}
\definecolor{YELLOW}{rgb}{1,1,0}
\definecolor{ORANGE}{rgb}{1,0.647,0}
\definecolor{PURPLE}{rgb}{0.627,0.126,0.941}
\definecolor{PURPLE}{named}{purple}
\definecolor{PINK}{rgb}{1,0.412,0.706}
\definecolor{WHEAT}{rgb}{1,0.8,0.6}
\definecolor{BLUE}{rgb}{0,0,1}
\definecolor{GRAY}{named}{gray}
\definecolor{CYAN}{named}{cyan}

\newcommand{\orchid}[1]{\textcolor{Orchid}{#1}}
\newcommand{\defun}[1]{\orchid{#1}}

\newcommand{\BROWN}[1]{\textcolor{BROWN}{#1}}
\newcommand{\RED}[1]{\textcolor{red}{#1}}
\newcommand{\YELLOW}[1]{\textcolor{YELLOW}{#1}}
\newcommand{\PINK}[1]{\textcolor{PINK}{#1}}
\newcommand{\WHEAT}[1]{\textcolor{wheat}{#1}}
\newcommand{\GREEN}[1]{\textcolor{GREEN}{#1}}
\newcommand{\PURPLE}[1]{\textcolor{PURPLE}{#1}}
\newcommand{\BLACK}[1]{\textcolor{black}{#1}}
\newcommand{\WHITE}[1]{\textcolor{WHITE}{#1}}
\newcommand{\MAGENTA}[1]{\textcolor{MAGENTA}{#1}}
\newcommand{\ORANGE}[1]{\textcolor{ORANGE}{#1}}
\newcommand{\BLUE}[1]{\textcolor{BLUE}{#1}}
\newcommand{\GRAY}[1]{\textcolor{gray}{#1}}
\newcommand{\CYAN}[1]{\textcolor{cyan }{#1}}

\newcommand{\reference}[2]{\textcolor{PINK}{[#1~#2]}}
%\newcommand{\vect}[1]{\stackrel{\rightarrow}{#1}}

% Use some nice templates
\beamertemplatetransparentcovereddynamic

\newcommand{\A}{{\mathbb A}}
\newcommand{\degr}{\mathrm{deg}}

\title{A CLOS protocol for lexical environments}

\author{Robert Strandh\\Ir�ne Durand}
\date{March, 2022}

%\inputtex{macros.tex}


\begin{document}
\frame{
\titlepage
\vfill
\small{European Lisp Symposium, Porto, Portugal \hfill ELS2022}
}

\setbeamertemplate{footline}{
\vspace{-1em}
\hspace*{1ex}{~} \GRAY{\insertframenumber/\inserttotalframenumber}
}

\frame{
\frametitle{Context: The \sicl{} project}

https://github.com/robert-strandh/SICL

Several objectives:

\begin{itemize}
\item Create high-quality \emph{modules} for implementors of
  \commonlisp{} systems.
\item Improve existing techniques with respect to algorithms and data
  structures where possible.
\item Improve readability and maintainability of code.
\item Improve documentation.
\item Ultimately, create a new implementation based on these modules.
\end{itemize}
}

\frame{
\frametitle{Cleavir compiler framework}

\begin{itemize}
\item Used by \sicl{}.
\item First pass translates a concrete syntax tree (CST) to an
  abstract syntax tree (AST).
\item Equivalent to ``minimal compilation'' as defined by the
  \commonlisp{} standard.
\item Special case of a ``code walker''.
\item Needs to maintain a \emph{lexical compilation environment}.
\end{itemize}
}

\frame{
\frametitle{Lexical compilation environment}

\begin{itemize}
\item Reflects the nested structure of code.
\item Contains information about:
  \begin{itemize}
  \item variables (lexical, special, constant),
  \item functions,
  \item macros,
  \item symbol macros,
  \item blocks,
  \item \texttt{tagbody} tags,
  \item declarations that are not associated with functions or
    variables (in particular \texttt{optimize}).
  \end{itemize}
\end{itemize}
}

\frame{
\frametitle{Lexical compilation environment}

\begin{itemize}
\item Is passed as the second argument to every macro function.
\item When it is \texttt{nil}, the ``null lexical environment'' is
  designated, which is the same as the global environment. 
\item The \commonlisp{} standard does not define any operations on
  environment objects.
\item But ``Common Lisp the Language (second edition)'' (CLtL2) has a
  section with such operators.
\end{itemize}
}

\frame{
\frametitle{CLtL2 protocol}
Function \texttt{variable-information}
\vskip 0.25cm
Returns information about a name in a variable position.
\vskip 0.25cm
\begin{itemize}
\item Arguments: A symbol and an optional environment.
\item Values:
  \begin{enumerate}
  \item type of binding (\texttt{:lexical}, \texttt{:special},
    \texttt{:symbol-macro}, \texttt{:constant}, or \texttt{nil}).
  \item A Boolean indicating whether the binding is local.
  \item Association list of declarations that apply to the binding.
  \end{enumerate}
\end{itemize}
}

\frame{
\frametitle{CLtL2 protocol}
Function \texttt{function-information}
\vskip 0.25cm
Returns information about a name in a function position.
\vskip 0.25cm
\begin{itemize}
\item Arguments: A function name and an optional environment.
\item Values:
  \begin{enumerate}
  \item type of binding (\texttt{:function}, \texttt{:macro},
    \texttt{:special-form}, or \texttt{nil}).
  \item A Boolean indicating whether the binding is local.
  \item Association list of declarations that apply to the binding.
  \end{enumerate}
\end{itemize}
}

\frame{
\frametitle{CLtL2 protocol}
Function \texttt{declaration-information}
\vskip 0.25cm
Returns information about declarations that do not apply to a
particular binding.  
\vskip 0.25cm
\begin{itemize}
\item Arguments: A declaration identifier and an optional environment.
\item Value:
  \begin{itemize}
  \item If the declaration identifier is \texttt{optimize}, then a
    list of entries of the form (\textit{quality value}).
  \item If the declaration identifier is \texttt{declaration}, then a
    list of declaration identifiers supplied to the
    \texttt{declaration} proclamation.
  \end{itemize}
\end{itemize}
}

\frame{
\frametitle{CLtL2 protocol}
Function \texttt{augment-environment}
\vskip 0.25cm
Given an environment, returns a new environment augmented with the
given information.
\vskip 0.25cm
Arguments:
\begin{enumerate}
\item An environment object.
\item Keyword arguments: \texttt{:variable}, \texttt{:symbol-macro},
  \texttt{:function}, \texttt{:macro}, \texttt{:declare}.
\end{enumerate}
}

\frame{
\frametitle{CLtL2 protocol}
Function \texttt{parse-macro}
\vskip 0.25cm
Given a macro definition, return a macro lambda expression. 
\vskip 0.25cm
Arguments:
\begin{enumerate}
\item \texttt{name}.  The name of the macro. 
\item \texttt{lambda-list}.  A macro lambda list.
\item \texttt{body}.  The macro body as a list of forms, etc.
\item \texttt{env}.  An optional environment.  Not sure what it is
  used for.
\end{enumerate}
}

\frame{
\frametitle{CLtL2 protocol}
Function \texttt{enclose}
\vskip 0.25cm
Given a macro lambda expression and an environment, return a macro function.
\vskip 0.25cm
Arguments:
\begin{enumerate}
\item \texttt{lambda-expression}.  A lambda expression, possibly
  created by \texttt{parse-macro}.
\item \texttt{env}.  An optional environment.
\end{enumerate}
}

\frame{
\frametitle{CLtL2 protocol}
It is incomplete:
\vskip 0.25cm
\begin{itemize}
\item No functionality for information about \texttt{block}s.
\item No functionality for information about \texttt{tabody}s.
\item No associated information for functions, variables, etc.
\end{itemize}
\vskip 0.25cm
It is not possible to extend in a compatible way, because
of multiple return values.
\vskip 0.25cm
No free \commonlisp{} implementation we investigated (\sbcl{},
\cmucl{}, \ecl{}, \ccl{}) uses the CLtL2 protocol for the native
compiler.
}

\frame{
\frametitle{Our solution}
\vskip 0.25cm
https://github.com/s-expressionists/Trucler
\vskip 0.25cm
\begin{itemize}
\item Have the query functions return instances of standard classes.
\item Define separate functions for each type of environment
  augmentation.
\end{itemize}
}

\frame{
\frametitle{Our solution}
\vskip 0.25cm
Query functions:
\begin{itemize}
\item \texttt{describe-variable}
\item \texttt{describe-function}
\item \texttt{describe-block}
\item \texttt{describe-tag}
\item \texttt{describe-optimize}
\item \texttt{describe-declarations}
\end{itemize}
}

\frame{
\frametitle{Our solution}
\vskip 0.25cm
Example: \texttt{describe-function}
\vskip 0.25cm
Parameters:
\begin{itemize}
\item \texttt{client}.  Trucler does not specialize to this parameter.
  Callers should supply an instance of a standard class.  Client code
  can define methods that specialize to their own client class(es).
\item \texttt{environment}.  Client code must supply an instance of a
  standard class, even to designate the global environment.
\item \texttt{name}
\end{itemize}
}

\frame{
\frametitle{Our solution}
\vskip 0.25cm
Example: \texttt{describe-function}
\vskip 0.25cm
Returns a subclass of one of:
\begin{itemize}
\item \texttt{function-description}\\
  Subclasses:
  \begin{itemize}
  \item \texttt{global-function-description}\\
    Subclass:
    \begin{itemize}
    \item \texttt{generic-function-description}
    \end{itemize}
  \item \texttt{local-function-description}
  \end{itemize}
\item \texttt{macro-description}\\
  Subclasses:
  \begin{itemize}
  \item \texttt{global-macro-description}
  \item \texttt{local-macro-description}
  \end{itemize}
\item \texttt{special-operator-description}
\end{itemize}
}

\frame{
\frametitle{Our solution}
\vskip 0.25cm
Example: \texttt{describe-function}
\vskip 0.25cm
Accessors for \texttt{global-function-description}:
\begin{itemize}
\item \texttt{name}
\item \texttt{type}
\item \texttt{inline}
\item \texttt{inline-data}
\item \texttt{ignore}
\item \texttt{dynamic-extent}
\end{itemize}
}


\frame{
\frametitle{Our solution}
\vskip 0.25cm
Augmentation functions:
\begin{itemize}
\item \texttt{add-lexical-variable}
\item \texttt{add-special-variable}
\item \texttt{add-local-symbol-macro}
\item \texttt{add-local-function}
\item \texttt{add-local-macro}
\item \texttt{add-block}
\item \texttt{add-tag}
\end{itemize}
}

\frame{
\frametitle{Our solution}
\vskip 0.25cm
Annotation functions:
\begin{itemize}
\item \texttt{add-variable-type}
\item \texttt{add-variable-ignore}
\item \texttt{add-variable-dynamic-extent}
\item \texttt{add-function-type}
\item \texttt{add-function-ignore}
\item \texttt{add-function-dynamic-extent}
\end{itemize}
}

\frame{
\frametitle{Future work}

\begin{itemize}
\item Stuff
\item More stuff
\end{itemize}
}


\frame{
\frametitle{Thank you}

Questions?
}

%% \frame{\tableofcontents}
%% \bibliography{references}
%% \bibliographystyle{alpha}

\end{document}
