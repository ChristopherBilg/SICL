\documentclass{acm_proc_article-sp}
\usepackage[utf8]{inputenc}

\def\inputfig#1{\input #1}
\def\inputtex#1{\input #1}
\def\inputal#1{\input #1}
\def\inputcode#1{\input #1}

\inputtex{logos.tex}
\inputtex{refmacros.tex}
\inputtex{other-macros.tex}

\begin{document}
\title{A CLOS Version of the CLtL2 Environment Protocol}
\numberofauthors{1}
\author{\alignauthor
Robert Strandh\\
\affaddr{University of Bordeaux}\\
\affaddr{351, Cours de la Libération}\\
\affaddr{Talence, France}\\
\email{robert.strandh@u-bordeaux1.fr}}

\toappear{Permission to make digital or hard copies of all or part of
  this work for personal or classroom use is granted without fee
  provided that copies are not made or distributed for profit or
  commercial advantage and that copies bear this notice and the full
  citation on the first page. Copyrights for components of this work
  owned by others than the author(s) must be honored. Abstracting with
  credit is permitted. To copy otherwise, or republish, to post on
  servers or to redistribute to lists, requires prior specific
  permission and/or a fee. Request permissions from
  Permissions@acm.org.

%  ELS '15, April 20 - 21 2015, London, UK
%  Copyright is held by the owner/author(s). Publication rights licensed to ACM.
%  ACM 978-1-4503-2931-6/14/08\$15.00.???
%  http://dx.doi.org/10.1145/2635648.2635656
}

\maketitle

\begin{abstract}
The concept of an \emph{environment} is mentioned in many places in
the \commonlisp{} standard, but the nature of the object is not
specified.  For the purpose of this paper, an environment is a mapping
from \emph{names} to \emph{meanings}.

In this paper, we propose a replacement for the environment protocol
documented in the book Common Lisp the Language, second edition.
Rather than returning multiple values as the functions in that that
protocol do, the protocol suggested in this paper is designed so that
functions return instances of standard classes.  Accessor functions on
those instances supply the information needed by a compiler or any
other \emph{code walker} application.

The advantage of our approach is that a protocol based on generic
functions and standard classes is easier to extend in
backward-compatible ways than the previous protocol, so that
implementations can suggest additional functionality on these objects.
Furthermore, \clos{} features such as auxiliary methods can be used on
these objects, making it possible to extend or override functionality
provided by the protocol, for implementation-specific purposes.
\end{abstract}

\category{D.3.4}{Programming Languages}{Processors}
[Code generation, Run-time environments]

\terms{Algorithms, Languages}

\keywords{\clos{}, \commonlisp{}, Environment}

\inputtex{sec-introduction.tex}
\inputtex{sec-previous.tex}
\inputtex{sec-our-method.tex}
\inputtex{sec-benefits.tex}
\inputtex{sec-conclusions.tex}

\bibliographystyle{abbrv}
\bibliography{environment-info}
\end{document}

%%  LocalWords:  sandboxing runtime
