\section{Disadvantages of our technique}

Compared to the protocol defined in CLtL2, our protocol probably
involves more memory allocation, or ``consing''.  Multiple values are
likely to be handled without memory allocation in most high-end
\commonlisp{} implementations, whereas our query functions return
standard objects which obviously need to be allocated.  Initializing
the slots of these standard objects also comes with an additional
cost.

Our protocol consists of generic functions, and in implementations
with a mediocre implementation of generic dispatch, our protocol can
require more resources for function calls.  Furthermore, the multiple
values returned by the CLtL2 protocol are likely transmitted to the
caller in registers or some other relatively direct location, whereas
the information returned by our query functions is present in slots of
the standard objects being returned.  Accessing this information
involves calling a slot reader, which involves another call to a
generic function.

However, we believe that the work done by the code walker of a
compiler is small compared to that of other compilation phases such as
optimization of intermediate code.
