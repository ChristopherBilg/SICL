\section{Our technique}

We define a \clos{}-based protocol for accessing and augmenting
lexical environment.  This protocol is defined and implemented in the
\trucler{} library.%
\footnote{https://github.com/s-expressionists/Trucler}

\subsection{Querying the environment}

\subsubsection{Client protocol}

The \emph{client protocol} defines the classes and operations that a
language processor such as a compiler uses in order to query and
augment the lexical environment according to the language elements
being processed.

\vskip -0.05cm
In order for our definitions to fit in a column, we have abbreviated
``Generic Function'' as ``GF''.
\vskip -0.05cm

{\footnotesize
\Defgeneric {describe-variable} {client environment name}
}

This function is called by the language processor whenever a symbol in
a \emph{variable} position is to be compiled.  If successful, it
returns an instance of one of the classes described in
\refSec{sec-instantiable-classes-variable-desciption}.

{\footnotesize
\Defcondition {no-variable-description}
}

This condition is signaled by \trucler{} when a client-supplied method
on the generic function \texttt{describe-variable} returns \texttt{nil}.

{\footnotesize
\Defmethod {name} {(condition {\tt no-variable-description})}
}

This method returns the name of the variable for which no description was
available.

\subsubsection{Function information}

{\footnotesize
\Defgeneric {describe-function} {client environment name}
}

This function is called by the language processor whenever a symbol in
a \emph{function} position is to be compiled or whenever a function
name is found in a context where it is known to refer to a function.
If successful, it returns an instance of one of the classes described
in \refSec{sec-instantiable-classes-function-desciption}.

{\footnotesize
\Defcondition {no-function-description}
}

This condition is signaled by \trucler{} when a client-supplied method
on the generic function \texttt{describe-function} returns \texttt{nil}.

{\footnotesize
\Defmethod {name} {(condition {\tt no-function-description})}
}

This method returns the name of the function for which no description was
available.

\subsubsection{Block information}

{\footnotesize
\Defgeneric {describe-block} {client environment name}
}

This function is called by the language processor whenever a symbol
referring to a \emph{block} is found, typically in a
\texttt{return-from} form.  If successful, it returns an instance of
the class described in
\refSec{sec-instantiable-classes-block-desciption}.

{\footnotesize
\Defcondition {no-block-description}
}

This condition is signaled by \trucler{} when a client-supplied method
on the generic function \texttt{describe-block} returns \texttt{nil}.

{\footnotesize
\Defmethod {name} {(condition {\tt no-block-description})}
}

This method returns the name of the block for which no description was
available.

\subsubsection{Tag information}

{\footnotesize
\Defgeneric {describe-tag} {client environment tag}
}

This function is called by the language processor whenever a symbol or
an integer referring to a \emph{tag} is found, typically in a
\texttt{go} form.  If successful, it returns an instance of the
class described in \refSec{sec-instantiable-classes-tag-desciption}.

{\footnotesize
\Defcondition {no-tag-description}
}

This condition is signaled by \trucler{} when a client-supplied method
on the generic function \texttt{describe-tag} returns \texttt{nil}.

{\footnotesize
\Defmethod {name} {(condition {\tt no-tag-description})}
}

This method returns the name of the tag for which no description was
available.

\subsubsection{Optimize information}

{\footnotesize
\Defgeneric {describe-optimize} {client environment}
}

Client-supplied methods on this function must always return a valid
instance of the class \texttt{optimize-description}.  It returns an
instance of the class described in
\refSec{sec-instantiable-classes-optimize-desciption}.

\subsubsection{Declarations information}

{\footnotesize
\Defgeneric {describe-declarations} {client environment}
}

Client-supplied methods on this function must always return a valid
instance of the class \texttt{declarations-description}.  It returns
an instance of the class described in
\refSec{sec-instantiable-classes-declarations-description}.
