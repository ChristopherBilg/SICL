\section{Our technique}

We define several \clos{}-based protocols for accessing and augmenting
lexical environment.  This protocol is defined and implemented in the
\trucler{} library.%
\footnote{https://github.com/s-expressionists/Trucler}

\subsection{Querying the environment}

A language processor calls one of the query functions in order to
determine the nature of a language element, depending on the position
in source code of that language element.  All these functions are
generic, and they all take a \texttt{client} parameter and an
\texttt{environment} parameter.  Methods defined by \trucler{} do not
specialize to the \texttt{client} parameter.  Client code should pass
an object specific to the application as a value of that parameter,
and it can supply methods specialized to the class of this object, for
the purpose of extending or overriding default behavior.  The
\texttt{environment} parameter is an object of the type used by the
implementation that \trucler{} is configured for.  Functions that are
used to query a particular \emph{name} have an additional parameter
for this purpose.

\begin{itemize}
\item \texttt{describe-variable}.  This function is called by the
  language processor when a name is found in a variable position.  It
  returns an instance of a class that distinguishes lexical variables,
  special variables, constant variables, and symbol macros.
\item \texttt{describe-function}.  This function is called by the
  language processor when a name is found in a function position.  It
  returns an instance of a class that distinguishes global functions,
  local functions, and macros.
\item \texttt{describe-block}.
\item \texttt{describe-tag}.
\item \texttt{describe-optimize}.
\item \texttt{describe-declarations}.
\end{itemize}
