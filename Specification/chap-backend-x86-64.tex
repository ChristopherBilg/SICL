\chapter{x86-64}
\label{chapter-backend-x86-64}

\section{Register usage}
\label{sec-backend-x86-84-register-use}

The standard calling conventions defined by the vendors, and used by
languages such as \clanguage{} use the registers as follows:

\begin{tabular}{|l|l|l|}
\hline
Name & Used for & Saved by\\
\hline
\hline
RAX & First return value & Caller\\
RBX & Optional base pointer & Callee\\
RCX & Fourth argument & Caller \\
RDX & Third argument, second return value & Caller\\
RSP & Stack pointer &\\
RBP & Frame pointer & Callee\\
RSI & Second argument & Caller\\
RDI & First argument & Caller\\
R8 & Fifth argument & Caller\\
R9 & Sixth argument & Caller\\
R10 & Temporary, static chain pointer & Caller\\
R11 & Temporary & Caller\\
R12 & Temporary & Callee\\
R13 & Temporary & Callee\\
R14 & Temporary & Callee\\
R15 & Temporary & Callee\\
\hline
\end{tabular}

We mostly respect this standard, and define the register allocation as
follows:

\begin{tabular}{|l|l|l|}
\hline
Name & Used for\\
\hline
\hline
RAX & First return value\\
RBX & Dynamic environment\\
RCX & Fourth argument, third return value \\
RDX & Third argument, second return value\\
RSP & Stack pointer \\
RBP & Frame pointer \\
RSI & Second argument, fourth return value\\
RDI & First argument, value count\\
R8  & Fifth argument\\
R9  &  Argument count, fifth return value\\
R10 & Static env. argument\\
R11 & Call-site descriptor\\
R12 & Register variable\\
R13 & Register variable\\
R14 & Register variable\\
R15 & Register variable\\
\hline
\end{tabular}

We do not use any callee-saves registers.  The call-site manager is
able to optimize register use for leaf routines, by essentially
treating every register as a callee-saves register, except that this
optimization is more flexible, since the call-site manager has
knowledge about both the caller and the callee.

By not using callee-saves registers, we are also able to simplify both
the register allocator, the compiler, and the root-finding phase of
the garbage collector.

We use the \texttt{FS} segment register to hold an instance of the
\texttt{thread} class.  Such an instance will be used in a variety of
situations including for multiple return values, debugging support,
etc.

For information about the call-site descriptor, see
\refSec{sec-call-site-descriptor}.

\section{Representation of function objects}

\begin{itemize}
\item A static environment.
\item The entry point of the function as a raw machine address.  Since
  entry points are word aligned, this value looks like a fixnum.
\end{itemize}

\section{Calling conventions}

\refFig{fig-x86-64-stack-frame} shows the layout of a stack frame.

\begin{figure}
\begin{center}
\inputfig{fig-x86-64-stack-frame.pdf_t}
\end{center}
\caption{\label{fig-x86-64-stack-frame}
Stack frame for the x86-64 backend.}
\end{figure}

Normal call to external function, passing at most $5$ arguments:

\begin{enumerate}
\item Compute the callee function object and the arguments into
  temporary locations.
\item Store the arguments in RDI, RSI, RDX, RCX, and R8.
\item Store the argument count in R9 as a fixnum.
\item Load the static environment of the callee from the callee
  function object into R10.
\item Load the call-site descriptor into R11.
\item Push the value of \texttt{RBP} on the stack.
\item Copy the value of RSP into \texttt{RBP}, establishing the
  (empty) stack frame for the callee.
\item Load the entry point address of the callee from the callee
  function object into an available scratch register, typically RAX.
\item Use the CALL instruction with that register as an argument,
  pushing the return address on the stack and transferring control to
  the callee.
\end{enumerate}

Normal call to external function, passing more than $5$ arguments:

\begin{enumerate}
\item Compute the callee function object and the arguments into
  temporary locations.
\item Subtract $8(N - 3)$ from RSP, where $N$ is the number of
  arguments to pass, thus leaving room in the callee stack frame for
  the $N - 5$ arguments, the return address, and the caller \texttt{RBP}.
\item Store the first $5$ arguments in RDI, RSI, RDX, RCX, and R8.
\item Store the remaining arguments in [RSP$ + 0$], [RSP$ + 8$],
  $\ldots$, [RSP$ + 8(N - 6)$] in that order, so that the sixth
  argument is on top of the stack.
\item Store the argument count in R9 as a fixnum.
\item Load the static environment of the callee from the callee
  function object into R10.
\item Load the call-site descriptor into R11.
\item Store the value of \texttt{RBP} into [RSP + 8(N - 4)]
\item Copy the value of RSP$ + 8(N - 4)$ into \texttt{RBP}, establishing the
  stack frame for the callee.  The instruction LEA can be used for
  this purpose.
\item Load the entry point address of the callee from the callee
  function object into an available scratch register, typically RAX.
\item Use the CALL instruction with that register as an argument,
  pushing the return address on the stack and transferring control to
  the callee.
\end{enumerate}

By using a \texttt{CALL}/\texttt{RET} pair instead of (say) the caller
storing the return address in its final place using some other method,
we make sure that the predictor for the return address of the
processor makes the right guess about the eventual address to be used.

\refFig{fig-x86-64-stack-frame-at-entry} shows the layout of the stack
upon entry to a function when more than $5$ arguments are passed.
Notice that the return address is not in its final place, and the
final place for the return address is marked ``unused'' in
\refFig{fig-x86-64-stack-frame-at-entry}.  Similarly, the call-site
descriptor is in register R11 and has not yet been copied to its final
place in the stack frame.

\begin{figure}
\begin{center}
\inputfig{fig-x86-64-stack-frame-at-entry.pdf_t}
\end{center}
\caption{\label{fig-x86-64-stack-frame-at-entry}
Stack frame at entry with more than 5 arguments.}
\end{figure}

Tail call to external function, passing at most $5$ arguments:

\begin{enumerate}
\item Compute the callee function object and the arguments into
  temporary locations.
\item Store the arguments in RDI, RSI, RDX, RCX, and R8.
\item Store the argument count in R9 as a fixnum.
\item Load the static environment of the callee from the callee
  function object into R10.
\item Copy the value of \texttt{RBP}$ - 8$ to RSP, establishing the stack frame
  for the callee, containing only the return address.  The LEA
  instruction can be used for this purpose.
\item Load the entry point address of the callee from the callee
  function object into an available scratch register, typically RAX.
\item Use the JMP instruction with that register as an argument,
  transferring control to the callee.
\end{enumerate}

Tail call to external function, passing more than $5$ arguments:%
\fixme{Determine the protocol.}

For \emph{internal calls} there is greater freedom, because the caller
and the callee were compiled simultaneously.  In particular, the
caller might copy some arbitrary \emph{prefix} of the code of the
callee in order to optimize it in the context of the caller.  This
prefix contains argument count checking and type checking of
arguments.  The address to use for the call is computed statically as
an offset from the current program counter, so that a CALL instruction
with a fixed relative address can be used.  Furthermore, the caller
might be able to avoid loading the static environment if it is known
that the callee uses the same static environment as the caller.

Upon function entry after an ordinary call, when more than $5$
arguments are passed, the callee must pop the return address off the
top of the stack and store it in its final location.  This can be done
with a single POP instruction, using [\texttt{RBP}$ - 8$] as the
destination.  When fewer than $5$ arguments are passed, the return
address is already in the right place.

Return from callee to caller, with no values:

\begin{enumerate}
\item Store NIL in RAX.
\item Store $0$ in RDI, represented as a fixnum.
\item Store the value of \texttt{RBP}$ - 8$ in RSP so that the stack frame
  contains only the return address.  To accomplish this effect, the
  callee can use the LEA instruction.
\item Return to the caller by executing the RET instruction.
\end{enumerate}

Return from callee to caller, with at least one value:

\begin{enumerate}
\item Store the first $5$ values to return in RAX, RDX, RCX, RSI, R9.
\item Store the remaining values in storage in the \texttt{thread}
  instance (contained in segment register \texttt{FS}) reserved for
  this purpose.
\item Store the number of values in RDI, represented as a fixnum.
\item Store the value of \texttt{RBP}$ - 8$ in RSP so that the stack frame
  contains only the return address.  To accomplish this effect, the
  callee can use the LEA instruction.
\item Return to the caller by executing the RET instruction.
\end{enumerate}

\section{Use of the dynamic environment}

The dynamic environment is an ordinary list of \emph{entries}
allocated on the stack rather than on the heap.  The head of the list
is pointed to by the register \texttt{RBX}.
\seesec{sec-backend-x86-84-register-use}

\Defclass {dynamic-environment-entry}

This class is the base class of all entry classes of the dynamic
environment.

\Defclass {exit-point-entry}

This class is a subclass of the class named
\texttt{dynamic-environment-entry}.  It is a superclass of all entries
that represent \emph{exit points}.  The \commonlisp{} standard is
somewhat unclear as to what constitutes and exit point.  The glossary
includes \texttt{unwind-protect} in the set of exit points.  However,
section 5.2 suggests otherwise.  Step 1 of the procedure described in
section 5.2 is to ``abandon'' all intermediate exit points.  But with
the definition in the glossary, the exit points representing
\texttt{unwind-protect} forms would also be ``abandoned'', whatever
that might mean.

Here, we restrict the term \emph{exit point} to be program points
established by \texttt{catch}, \texttt{block}, and \texttt{tagbody}.

\Defgeneric {valid-p} {exit-point-entry}

This generic function returns \textit{true} if and only if
\textit{exit-point-entry} is valid, i.e. if it has not been invalidated (or
``abandoned'') since its creation.

\Defgeneric {invalidate} {exit-point-entry}

This generic function invalidates the entry \textit{exit-point-entry}.
After this function has been called, a call to \texttt{valid-p} with
\textit{exit-point-entry} as an argument always returns \textit{false}.

\Defclass {special-binding-entry}

Instances of this class are used to represent the binding of a special
variable.  It is a subclass of the class named
\texttt{dynamic-environment-entry}.

\Definitarg {:symbol}

The value of this initarg is the symbol representing the special
variable to be bound.

\Definitarg {:value}

The value of this initarg is the value to which the special variable
is to be bound.

\Defgeneric {symbol} {special-binding-entry}

This generic function returns the symbol passed as the value of the
initarg \texttt{:symbol} when the entry \textit{special-binding-entry}
was created.

\Defgeneric {value} {special-binding-entry}

This generic function returns the current value of the special
variable represented by \textit{special-binding-entry}.

\Defgeneric {(setf value)} {new-value special-binding-entry}

This generic function sets the current value of the special
variable represented by \textit{special-binding-entry}.

\Defclass {catch-entry}

An instance of this class is used to represent a \texttt{catch} tag.
It is a subclass of the class named \texttt{exit-point-entry}.

\Definitarg {:tag}

The value of this initarg is the \texttt{catch} tag represented by
this entry.

\Defgeneric {tag} {catch-entry}

This generic function returns the value of the \texttt{:tag} initarg
that was given when \textit{catch-entry} was created.

\Defclass {block/tagbody-entry}

An instance of this class is used to represent an exit point created
by \texttt{block} or \texttt{tagbody}.  It is a subclass of the class
named \texttt{exit-point-entry}.

\Definitarg {:identifier}

The value of this initarg is a unique identifier for this entry.  This
identifier becomes part of the static environment of any function
nested inside the \texttt{block} or \texttt{tagbody} form that
contains a \texttt{return-from} or a \texttt{go} to this form.

\Defgeneric {identifier} {block/tagbody-entry}

The value returned by a call to this generic function is the
identifier of the \texttt{block/tagbody-entry} as given by the initarg
\texttt{:identifier} when the entry was created.

\Defclass {unwind-protect-entry}

This class is a subclass of the class named
\texttt{dynamic-environment-entry}.

\Definitarg {:cleanup-thunk}

The value of this initarg is a thunk that encapsulates the cleanup
forms of the \texttt{unwind-protect} form.

\Defgeneric {cleanup-thunk} {unwind-protect-entry}

A call to this generic function returns the cleanup thunk of
\textit{unwind-protect-entry} that was supplied as the value of the
initarg \texttt{:cleanup-thunk} when the entry was created.

\Defclass {multiple-values-entry}

This class is a subclass of the class named
\texttt{dynamic-environment-entry}.  It is used to store a sequence of
multiple values when the registers and stack entries for this purpose
are insufficient.  In particular, \texttt{multiple-value-prog1} and
\texttt{multiple-value-call} may need one or more of these entries.

\texttt{catch} is implemented as a call to a function.  This function
establishes a \texttt{catch} tag and calls a thunk containing the body
of the \texttt{catch} form.

\texttt{throw} searches the dynamic environment for an entry with the
right \texttt{catch} tag which is also valid.  The point to which
control is to be transferred is stored as the return value of the
stack frame containing the \texttt{catch} tag.

A \texttt{block} form may establish an exit point.  In the most
general case, a \texttt{return-from} is executed from a function
lexically-enclosed inside the \texttt{block} with an arbitrary number
of intervening stack frames.  When this is the case, upon entry to the
\texttt{block} form, a \texttt{block}/\texttt{tagbody} entry with a
fresh identifier is established.  When a \texttt{return-from} is
executed, the point to which control is to be transferred is known
statically.  The identifier is also stored in a lexical variable in
the static environment of closures established inside the
\texttt{block} form that may execute a corresponding
\texttt{return-from} form.  When the \texttt{block} form is exited normally, the
\texttt{block}/\texttt{tagbody} entry is popped off the stack and off
the dynamic environment.

\texttt{return-from} accesses the identifier from the static
environment and then searches the dynamic environment for a
corresponding identifier in a \texttt{block}/\texttt{tagbody} entry.
If one with the right identifier is found, the lexical environment is
restored, and control is transferred to the statically known address.

A \texttt{tagbody} may establish several exit points.  In the most
general case, a \texttt{go} is executed from a function
lexically-enclosed inside the \texttt{tagbody} with an arbitrary
number of intervening stack frames.  When this is the case, upon entry
to the \texttt{tagbody} form, a \texttt{block}/\texttt{tagbody} entry
with a fresh identifier is established.  When a \texttt{go} is
executed, the point to which control is to be transferred is known
statically.  The identifier is also stored in a lexical variable in
the static environment of closures established inside the
\texttt{tagbody} form that may execute a corresponding \texttt{go}
form.  When the \texttt{tagbody} form is exited normally, the
\texttt{block}/\texttt{tagbody} entry is popped off the stack and off
the dynamic environment.

\texttt{go} accesses the identifier from the static environment and
then searches the dynamic environment for a corresponding identifier
in a \texttt{block}/\texttt{tagbody} entry.  If one with the right
identifier is found, the lexical environment is restored, and control
is transferred to the statically known address.

\section{Transfer of control to an exit point}

Whenever transfer of control to an exit point is initiated, the exit
point is first searched for.  If no valid exit point can be found, an
error is signaled.  If a valid exit point is found, the stack must
then be unwound.  First, the dynamic environment is traversed for any
intervening exit points, and they are marked as invalid as indicated
above.  Traversal stops when the stack frame of the valid exit point
is reached.  Unwinding now begins.  The dynamic environment is
traversed again and thunks in \texttt{unwind-protect} entries are
executed.  The traversal again stops when the stack frame of the valid
exit point is reached.

\section{Address space layout}

The architecture specification guarantees that the available address
space is at least $2^{48}$ bytes, with half of it at the low end of
the full potential $2^{64}$ byte address space, and the other half at
the high end of the full potential address space.  The upper half is
typically reserved for the operating system, so for applications only
the low $2^{47}$ bytes are available.  We divide this address space as
follows:

\begin{itemize}
\item The space reserved for dyads in the global heap (see
  \refApp{app-memory-allocator}) starts at address $0$ and ends at
  $2^{45}$.  Only a small part of it is initially allocated, of course
  and it grows as needed.
\item The space reserved for racks in the global heap (see
  \refApp{app-memory-allocator}) starts at address $2^{45}$ and ends at
  $2^{46}$.  Only a small part of it is initially allocated, of course
  and it grows as needed.
\item Each thread is given an address space of $2^{30}$ bytes with the
  low part (a few megabytes) reserved for the nursery and the rest
  reserved for the stack.  Only a small part of the stack is initially
  allocated, and it grows as needed.  The address spaces for threads
  start at address $2^{46}$.
\end{itemize}

This layout is illustrated in \refFig{fig-address-space-x86-64}.

\begin{figure}
\begin{center}
\inputfig{fig-address-space-x86-64.pdf_t}
\end{center}
\caption{\label{fig-address-space-x86-64}
Address space layout for the x86-64 backend.}
\end{figure}

\section{Parsing keyword arguments}

When the callee accepts keyword arguments, it is convenient to have
all the arguments in a properly-ordered sequence somewhere in memory.
We obtain this sequence by pushing the register arguments to the stack
in reverse order, so that the first argument is at the top of the
stack.  When more than $5$ arguments are passed, the program counter
is popped off the top of the stack, thereby moving it to its final
destination before the register arguments are pushed.

%%  LocalWords:  callee
