\chapter{Compiler infrastructure -- \cleavir{}}

\cleavir{} is a collection of modules that can be used to build a
\commonlisp{} compiler.  It currently contains modules for the
following functionalities:

\begin{itemize}
\item Definitions of classes and generic functions for representing
  and manipulating abstract syntax trees (ASTs).  An AST can be seen
  as a representation (as standard objects) of source code that has
  been \emph{minimally compiled}%
  \footnote{See section 3.2.2.2 of the \commonlisp{} standard},
  meaning that macros and compiler macros have been expanded, so that
  the meaning of the AST no longer depends on a particular
  environment.
\item Definitions of generic functions for translating a concrete
  syntax tree (see \refChap{chap-concrete-syntax-tree}) to an abstract
  syntax tree.
\item Definitions of classes and generic functions for representing
  and manipulating intermediate code in the form of a \emph{graph of
    instructions}, similar to the intermediate code used in compilers
  for more traditional languages.
\item Definitions of generic functions for translating an abstract
  syntax tree to an instruction graph.
\end{itemize}

Currently, two versions of \cleavir{} exists.

The more advanced version has been extracted to a separate repository%
\footnote{https://github.com/s-expressionists/Cleavir}, and is
maintained and improved by Alex Wood and Charles Zhang.  It is being
used in the main compiler of the \clasp{} implementation of
\commonlisp{}.

A more primitive version is located in the directory
\texttt{Code/Cleavir} of the \sysname{} repository.  The main reason
for the existence of this module is to provide a stable basis for the
\sysname{} compiler while \sysname{} is still frequently modified in
many ways.  The ultimate goal is to abandon this version in favor of
the extracted one.
