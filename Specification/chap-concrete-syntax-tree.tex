\chapter{Concrete syntax tree}
\label{chap-concrete-syntax-tree}

The Concrete Syntax Tree%
\footnote{https://github.com/s-expressionists/Concrete-Syntax-Tree}
module was designed to solve the problem of \emph{source tracking},
i.e., the association of some source code with some concept of
\emph{location} of that code in a source file.

This module handles the problem by wrapping S-expressions in standard
objects that can contain this additional information.  The standard
object in question mimics two kinds of S-expressions, namely
\texttt{cons} cells and \emph{atoms}.  Operations are provided by this
module to take the \texttt{first} and the \texttt{rest} of a standard
object that representes a \texttt{cons} cell, and to return the
\emph{raw} contents of a concrete syntax tree, i.e., the underlying
S-expression.

Furthermore, this module provides a certain number of higher-level
utilities for manipulating concrete syntax trees.  For example, it
contains a lambda-list parser for lambda lists that are represented as
concrete syntax trees.

A particularly important utility is one that \emph{reconstructs} a
concrete syntax tree from two arguments, namely a concrete syntax tree
and an S-expression.  This utility can be used in a compiler for macro
expansion.  \commonlisp{} macro expansion is specified to work on
S-expressions.  The problem, then, is to preserve source information
across macro expansion.  A compiler would then take a concrete syntax
tree, retrieve the raw underlying S-expression and pass that
S-expression to the macro function.  The macro function returns
another S-expression that typically contains some nodes of the
original S-expression and some nodes that were introduced by the macro
function.  The reconstruction utility tries to match nodes in the
output to nodes in the input, and construct a concrete syntax tree
from the result.  For \texttt{cons} cells, the task is fairly simple.
If the \emph{same} (as determined by \texttt{eq}) \texttt{cons} cell
exists in the input and in the output, the concrete syntax tree for
that \texttt{cons} cell in the input is reused in the reconstructed
concrete syntax tree.  For atoms, the problem is harder.  The concrete
syntax tree module uses some heuristics to provide a plausible
reconstruction.
