\chapter{How to contribute}

In this chapter, we give some guidelines to people who think they
might want to contribute to \sysname{}.  Specifically, we elaborate on
the main objectives with \sysname{}, which will then dictate what kind
of contributions we are interested in, and how to determine whether a
contribution is worthwhile to work on.

The important objectives of \sysname{} are (in no particular order):

\begin{itemize}
\item Simplicity.  We want the design to be as simple as possible.  We
  tolerate exceptions to this rule only if there is some major
  advantage to them, such as performance.  For example objects of
  type \texttt{cons}, \texttt{fixnum}, \texttt{character}, and
  \texttt{single-float} are not standard objects, because we are
  convinced that it would be hard to get good performance if they
  were.
\item Maintainability.  We want the code to be readable and
  understandable.  This objective implies not only that the code
  should have comments explaining the techniques being used, but also
  that the code should be idiomatic, divided into reasonable-sized
  units with appropriate names, and respect the guidelines in
  \refChap{chap-general-commonlisp-style-guide}, and
  \refChap{sicl-specific-style-guide}.
\item Debuggability.  We want \sysname{} to be the preferred
  \commonlisp{} implementation for debugging application code.  This
  objective implies that error messages should be excellent, and they
  should refer to source locations whenever possible.  But it also
  means that we must be able to set breakpoints for the purpose of
  pausing the execution and examining variable values.
\item Features.  We are obviously interested in \sysname{} having the
  features that application programmers expect.  However, there are
  some features that we are not particularly interested in, because we
  do not intend to use them ourselves, and they would likely increase
  our own maintenance burden.  Features like that would be better to
  implement as separate repositories that interested users could add
  if they like.  In particular, we are not interested in maintaining a
  foreign-function interface (FFI).
\item Performance.  Performance is a major objective for \sysname{}.
  However, we do not want maximum performance at all cost.  Other
  objectives must be respected as well.  In particular, we want to try
  to avoid code for handling specific optimization situations if there
  is a more general optimization technique that can handle several
  such situations.  For example, while some \commonlisp{}
  implementation can avoid checking the argument count by having
  several special entry points in each function, and avoid keyword
  arguments by introducing many compiler macros, our technique for
  call-site optimization is more general, and subsumes these other
  techniques, as well as some others.
\end{itemize}
