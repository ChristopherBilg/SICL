\chapter{Sequence functions}

This module was written entirely by Marco Heisig.  It provides
high-performance implementations of the functions in the ``sequences''
chapter of the \hs{}.  High performance is obtained by identifying
important special cases such as the use of \texttt{:test} function
\texttt{eq}, or \texttt{equal}, or the use of a \texttt{:key} of
\texttt{identity}.  These special cases are handled by macros
according to the technique described by our 2017 ELS paper
\cite{Durand:2017:ELS:Sequence}.

In addition to the technique described in that paper, Marco Heisig
decided to write the sequence functions as generic functions,
specialized to the type of the sequence argument.  Many
implementations have specialized versions of vectors, based on element
type, and a method specialized this way can often be significantly
faster than code that uses a generic \texttt{vector} type.  In order
to account for the different set of vector subclasses available in
different \commonlisp{} implementations, a macro
\texttt{replicate-for-each-vector-class} is used to generate a method
for each such subclass.  Client code can customize this module by
defining this macro according to its set of vector subclasses.

This module can be used as an ``extrinsic'' module, i.e., it can be
loaded into an existing \commonlisp{} implementation without
clobbering the native sequence functions of that implementation.  This
feature has been used to compare the performance of the functions in
this module to that of the native sequence functions of \sbcl{}, and
the result is very encouraging, in that many functions in this module
are as fast, or faster, than the native \sbcl{} equivalents.

\section{Future work}
\label{sec-sequence-functions-future-work}

Concerning the \emph{sorting functions} (i.e., \texttt{sort} and
\texttt{stable-sort}) there is an interesting challenge with respect
to finding a stable sorting algorithm for vectors that uses little
extra space.  The naive version of mergesort uses $O(n)$ extra space,
but some research (\cite{Huang:1990:FSM:898863},
\cite{Huang:1988:PIM:42392.42403},
\cite{Katajainen:1996:PIM:642136.642138}) indicates that it is
possible to obtain an in-place stable version of mergesort.  Since
mergesort is typically significantly faster than quicksort, this would
be a worthwhile direction to pursue.

Currently, this module is located in the \texttt{Code/Sequence}
directory of the \sysname{} repository, but we may extract it to a
separate repository in the future.

%%  LocalWords:  subclasses
