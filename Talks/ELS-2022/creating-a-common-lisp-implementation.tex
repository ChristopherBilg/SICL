\documentclass{beamer}
\usepackage[utf8]{inputenc}
\beamertemplateshadingbackground{red!10}{blue!10}
%\usepackage{fancybox}
\usepackage{epsfig}
\usepackage{verbatim}
\usepackage{url}
%\usepackage{graphics}
%\usepackage{xcolor}
\usepackage{fancybox}
\usepackage{moreverb}
%\usepackage[all]{xy}
\usepackage{listings}
\usepackage{filecontents}
\usepackage{graphicx}

\lstset{
  language=Lisp,
  basicstyle=\scriptsize\ttfamily,
  keywordstyle={},
  commentstyle={},
  stringstyle={}}

\def\inputfig#1{\input #1}
\def\inputeps#1{\includegraphics{#1}}
\def\inputtex#1{\input #1}

\inputtex{logos.tex}

%\definecolor{ORANGE}{named}{Orange}

\definecolor{GREEN}{rgb}{0,0.8,0}
\definecolor{YELLOW}{rgb}{1,1,0}
\definecolor{ORANGE}{rgb}{1,0.647,0}
\definecolor{PURPLE}{rgb}{0.627,0.126,0.941}
\definecolor{PURPLE}{named}{purple}
\definecolor{PINK}{rgb}{1,0.412,0.706}
\definecolor{WHEAT}{rgb}{1,0.8,0.6}
\definecolor{BLUE}{rgb}{0,0,1}
\definecolor{GRAY}{named}{gray}
\definecolor{CYAN}{named}{cyan}

\newcommand{\orchid}[1]{\textcolor{Orchid}{#1}}
\newcommand{\defun}[1]{\orchid{#1}}

\newcommand{\BROWN}[1]{\textcolor{BROWN}{#1}}
\newcommand{\RED}[1]{\textcolor{red}{#1}}
\newcommand{\YELLOW}[1]{\textcolor{YELLOW}{#1}}
\newcommand{\PINK}[1]{\textcolor{PINK}{#1}}
\newcommand{\WHEAT}[1]{\textcolor{wheat}{#1}}
\newcommand{\GREEN}[1]{\textcolor{GREEN}{#1}}
\newcommand{\PURPLE}[1]{\textcolor{PURPLE}{#1}}
\newcommand{\BLACK}[1]{\textcolor{black}{#1}}
\newcommand{\WHITE}[1]{\textcolor{WHITE}{#1}}
\newcommand{\MAGENTA}[1]{\textcolor{MAGENTA}{#1}}
\newcommand{\ORANGE}[1]{\textcolor{ORANGE}{#1}}
\newcommand{\BLUE}[1]{\textcolor{BLUE}{#1}}
\newcommand{\GRAY}[1]{\textcolor{gray}{#1}}
\newcommand{\CYAN}[1]{\textcolor{cyan }{#1}}

\newcommand{\reference}[2]{\textcolor{PINK}{[#1~#2]}}
%\newcommand{\vect}[1]{\stackrel{\rightarrow}{#1}}

% Use some nice templates
\beamertemplatetransparentcovereddynamic

\newcommand{\A}{{\mathbb A}}
\newcommand{\degr}{\mathrm{deg}}

\title{Creating a \commonlisp{} implementation}

\author{Robert Strandh}
\date{January, 2022}

%\inputtex{macros.tex}


\begin{document}
\frame{
\titlepage
\vfill
\small{European Lisp Symposium 2022}
}

\setbeamertemplate{footline}{
\vspace{-1em}
\hspace*{1ex}{~} \GRAY{\insertframenumber/\inserttotalframenumber}
}

\frame{
\frametitle{Motivation}
\vskip 0.25cm
\begin{itemize}
\item Dissatisfaction with the way current \commonlisp{}
  implementations are written.
\item Duplication of system code between implementations.
\item Some such instances are justified.  Most are not.
\end{itemize}
}

\frame{
\frametitle{Initial idea}
\vskip 0.25cm
Create a set of \emph{modules} that can be used to create a complete
\commonlisp{} implementation from a minimal \emph{core}.
\vskip 0.25cm
Problem: Such modules must be ordered by dependency, and each one
written in a subset defined by preceding modules, creating a
maintenance nightmare. 
}

\frame{
\frametitle{Initial modules}
\vskip 0.25cm
\begin{itemize}
\item \texttt{format}
\item \texttt{loop}
\end{itemize}
\vskip 0.25cm
Both use generic functions and standard classes.
}

\frame{
  \frametitle{Compiler framework (Cleavir)}
\vskip 0.25cm
\begin{itemize}
\item Creation of an Abstract Syntax Tree (AST) from a Concrete Syntax
  Tree (CST) to
\item Creation of High-level Intermediate Representation (HIR) from an
  AST.  HIR is a traditional flow graph, with the restriction that every
  variable contains a \commonlisp{} object. 
\item Creation of Medium-level Intermediate Representation (MIR) from
  HIR.  MIR introduces explicit memory operations, so some objects are
  raw addresses and raw integers.
\item Creation of Low-level Intermediate Representation (LIR) from
  MIR.  LIR introduces registers, so it is backend specific.
\item Register allocation.
\item Code generation.
\end{itemize}
}

\frame{
  \frametitle{New techniques}
\vskip 0.25cm
\begin{itemize}
\item First-class global environments.
\item Fast generic dispatch.
\item Handling \texttt{:from-end} better.
\item Macros for simplifying sequence functions.
\item Garbage collector.
\item Debugging support.
\end{itemize}
}

\frame{
\frametitle{Current state}
\begin{itemize}
\item Bootstrapping mostly working.
\item \texttt{eval} semantics is used during bootstrapping.  Change to
  file-compilation semantics.
\item Different external modules have different conflicting
  requirements.  Fix by using one first-class global environment per
  external module during bootstrapping.
\item Simple call-site manager still not written.
\end{itemize}

}

\frame{
\frametitle{Future work}
\vskip 0.25cm
Extract more modules into separate repositories:
\vskip 0.25cm
\begin{itemize}
\item \texttt{format}
\item \texttt{loop}
\item \texttt{sequence functions}
\item \texttt{high-level list functions}
\item \texttt{hash tables?}
\end{itemize}
}

\frame{
\frametitle{Future work}
\vskip 0.25cm
(Research) Investigate the relationship between
static-single-assignment form (SSA) and global value numbering (GVN).
We think the latter might be strictly more general than the former.
\vskip 0.25cm
\begin{itemize}
\item Have the register allocator work on the result of global value
  numbering rather than on lexical locations.
\item Allow the register allocator to do rematerialization.
\item Compare with Cliff Click's ``sea of nodes'' representation.
\end{itemize}
}

\frame{
\frametitle{Collaborators}
\begin{itemize}
\item Alex Wood (Cleavir, CST)
\item Charles Zhang (Cleavir, CST)
\item Jan Moringen (Eclector, CST)
\item Marco Heisig (HIR evaluator, sequence functions, Trucler, CST)
\item Hayley Patton (hash tables)
\item Daniel Kochmanski (Clostrum)
\item lonjil (Incless)
\item Henry Harrington (Cyclosis)
\item Tarn Burton (Inravina)
\end{itemize}
}

\frame{
\frametitle{Thank you}
}

%% \frame{\tableofcontents}
%% \bibliography{references}
%% \bibliographystyle{alpha}

\end{document}
